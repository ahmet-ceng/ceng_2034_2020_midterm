\documentclass[onecolumn]{article}
%\usepackage{url}
%\usepackage{algorithmic}
\usepackage[a4paper]{geometry}
\usepackage{datetime}
\usepackage[margin=2em, font=small,labelfont=it]{caption}
\usepackage{graphicx}
\usepackage{mathpazo} % use palatino
\usepackage[scaled]{helvet} % helvetica
\usepackage{microtype}
\usepackage{amsmath}
\usepackage{subfigure}
% Letterspacing macros
\newcommand{\spacecaps}[1]{\textls[200]{\MakeUppercase{#1}}}
\newcommand{\spacesc}[1]{\textls[50]{\textsc{\MakeLowercase{#1}}}}

\title{\spacecaps{Assignment Report 1: Process and Thread Implementation}\\ \normalsize \spacesc{CENG2034, Operating Systems} }

\author{Ahmet Oral\\ahmetoral@posta.mu.edu.tr\\https://github.com/ahmet\-ceng/ceng\_2034\_2020\_midterm}
%\date{\today\\\currenttime}
\date{\today}

\begin{document}
\maketitle




\section{Introduction}
In this assignment I practice some basics about operating systems.I coded a python program to print required parameters and check url's.My goal is to learn and practice as much as I could,while coding required script.
\section{Assignments}
My objective was to write a script that; checks url's and prints PID,loadavg,cpu core count.Also program should close itself if loadavg value is near to the cpu core count.I stared by researching what I didn't know,then step by step I implemented what I needed to.
\subsection{Assignment}
I coded required values as shown in the photos below.

\begin{figure}[h]
\includegraphics[width=10cm]{report ahmet oral/OS_Code1.png}
\centering
\end{figure}

\begin{figure}[h]
\includegraphics[width=10cm]{report ahmet oral/OS_Code2.png}
\centering
\end{figure}



\section{Results}
\begin{figure}[h]
\includegraphics[width=10cm]{report ahmet oral/OS Output.png}
\centering
\end{figure}

As shown in the photo,program checks url's and prints if it's working or not.Then program prints PID,load averages and nproc.After that we see there is a note that says "Program will end itself if: nproc - min loadavg \<1" and it keeps working.5 minute loadavg is constantly changing and program will keep working unless those 2 values are close to each other.There is a comment in the second while loop saying \#print(load5).I put this command to check if my code is working and how the load5 changes over time.If you decide to delete the \# in that line you can see the 5 mi load avg value changing over and over (outputs per second becomes crazy tho :D ).


\section{Conclusion}    
Most important lesson I learned from this project is reading all of the assignment carefully.At first I didn't notice the "*Implement this with threads" at the end of part 4 :D ,so I had to change my code.If I didn't realised that I would get less point or if project was complicated and hard maybe I wouldn't have enough time to chance it.Besides that the project was fun to do and because I used linux, I got chance practiced some of the things I forget.Also now I know I can check url's and take some basic system information.All in all this assignment contributed new skills to me and I enjoyed doing it.


\nocite{*}
\bibliographystyle{plain}
\bibliography{references}
\end{document}

